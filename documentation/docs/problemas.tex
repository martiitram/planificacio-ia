\section{Descripción del modelado de los problemas}

\subsection{Objetos}
Los objetos que declaramos al inicio de un problema son, en función de su
tipo:
\begin{description}
  \item[Tipo \texttt{primero}:] 
    Todos los objetos que 
    son primer plato
    como por ejemplo, \texttt{paella}.
  \item[Tipo \texttt{segundo}:] 
    Todos los objetos que 
    son segundo plato
    como por ejemplo, \texttt{entrecot}.
  \item[Tipo \texttt{dia}:] 
    Todos los objetos que 
    son un dia de la semana
    como por ejemplo, \texttt{lunes}.
\end{description}

\subsection{Estado inicial}

El estado inicial se compone de una serie de inicializaciones de
predicados y funciones.

\paragraph{Inicialización de predicados}

En la inicialización, se inicializa un subconjunto de los
predicados del dominio. Enunciados a continuación, para cada
uno de ellos se declara lo siguiente:

\begin{itemize}
  \item 
    Los dias que son consecutivos a otros 
    mediante el predicado \texttt{consecutivo}.
    Todos los dias deben tener un consecutivo excepto el último día.
  \item 
    Cuál es el ultimo dia de la semana
    mediante el predicado \texttt{ultimo\_dia}.
    Es necesario que como mínimo uno de los dias de la semana sea el último.
  \item 
    Los platos que son incompatibles entre ellos
    mediante el predicado \texttt{incompatible}.
    No es necesario inicializar este predicado para que el programa se
    ejecute correctamente.
  \item 
    De que tipo es cada plato
    mediante el predicado \texttt{es\_de\_tipo}.
    Es necesario que todos los platos tengan un tipo.
  \item
    Si se quiere fijar un primer plato a un dia de la semana, es necesario
    instanciar los siguientes predicados para el dia deseado \texttt{dia},
    el plato deseado \texttt{plato} y el tipo del plato deseado
    \texttt{tipo}.

    \begin{itemize}
      \item \texttt{(servido\_primero dia)}
      \item \texttt{(menu\_primero dia plato)}
      \item \texttt{(tipo\_dia\_primero dia tipo)}
      \item \texttt{(preparado plato)}
    \end{itemize}

    Análogamente, para fijar un segundo:

    \begin{itemize}
      \item \texttt{(servido\_segundo dia)}
      \item \texttt{(menu\_segundo dia plato)}
      \item \texttt{(tipo\_dia\_segundo dia tipo)}
      \item \texttt{(preparado plato)}
    \end{itemize}

\end{itemize}

\paragraph{Inicialización de funciones}

De forma similar, se inicializan algunas funciones del dominio en las cuales
se declara lo siguiente:
\begin{itemize}
  \item 
    El numero de calorias que tiene un plato con la función 
    \texttt{calorias}.
    Es necesario que todos los platos tengan un numero de calorias asociado.
  \item 
    El precio de un plato con la función
    \texttt{precio\_plato}.
    Es necesario que todos los platos tengan un precio asociado.
  \item 
    El precio total del menú semanal con la función
    \texttt{precio\_total}.
    Esta funcion se debe inicializar a 0 para evitar un comportamiento 
    inesperado.
\end{itemize}

\subsection{Estado final}
